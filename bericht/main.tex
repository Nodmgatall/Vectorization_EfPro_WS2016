\documentclass[
	12pt,
	a4paper,
	BCOR10mm,
	%chapterprefix,
	DIV14,
	headsepline,
	usegeometry,
	%twoside,
	%openright
]{scrreprt}

\KOMAoptions{
	listof=totoc,
	bibliography=totoc,
	index=totoc
}

\usepackage[T1]{fontenc}
\usepackage[utf8]{inputenc}

\usepackage{lmodern}

\usepackage[ngerman,english]{babel}

\usepackage[toc]{appendix}
\usepackage{color}
\usepackage{eurosym}
\usepackage{fancyhdr}
\usepackage{geometry}
\usepackage{graphicx}
\usepackage[htt]{hyphenat}
\usepackage{listings}
\usepackage{lstautogobble}
\usepackage{microtype}
\usepackage[list=true,hypcap=true]{subcaption}
\usepackage{textcomp}
\usepackage{units}

\usepackage{varioref}
\usepackage[hidelinks]{hyperref}
\usepackage[capitalise,noabbrev]{cleveref}

\definecolor{uhhred}{cmyk}{0,1,1,0}

\lstset{
	basicstyle=\ttfamily,
	frame=single,
	numbers=left,
	language=C,
	breaklines=true,
	breakatwhitespace=true,
	postbreak=\hbox{$\hookrightarrow$ },
	showstringspaces=false,
	autogobble=true,
	upquote=true,
	tabsize=4,
	captionpos=b,
	morekeywords={int8_t,uint8_t,int16_t,uint16_t,int32_t,uint32_t,int64_t,uint64_t,size_t,ssize_t,off_t,intptr_t,uintptr_t,mode_t}
}

\makeatletter
\renewcommand*{\lstlistlistingname}{List of Listings}
\makeatother

\begin{document}

\newgeometry{left=2cm, top=2cm, right=2cm, bottom=2cm}

\begin{titlepage}
	\includegraphics[width=0.5\textwidth]{UHH-Logo_2010_Farbe_CMYK}

	\begin{center}
		{\Large \textcolor{uhhred}{\textbf{Report}}\par}

		\vspace{1cm}

		{\titlefont\huge Vectorization \par}

		\vspace{1cm}

		{\large vorgelegt von\par}

		\vspace{0.5cm}

		{\large Vorname Nachname\par}
	\end{center}

	\vfill

	{\large\noindent\begin{tabular}{l}
		Fakultät Informatik und Naturwissenschaften\\
		Fachbereich Informatik\\
		Arbeitsbereich Wissenschaftliches Rechnen
	\end{tabular}\par}

	\vspace{1cm}

	{\large\noindent\begin{tabular}{ll}
		Studiengang:    &  SSE\\
		Matrikelnummer: & 6420331\\
		\\
		Betreuer:       & Nabeeh Jumah \\
		\\
		Hamburg, 2017-03-25
	\end{tabular}\par}
\end{titlepage}

\restoregeometry

\chapter*{Abstract}

\thispagestyle{empty}

This report discusses the basics of vectorization. It contains basic examples as well as some problems a programmer will encounter when working with
vectorization. It includes some the different instruction sets as well as differences between
compilers. The goal of this is to give a robust basic knowledge of the principles of
vectorization and to give an introduction on how to apply this form of optimization.
%TODO: add: different approaches (intinsics, auto vectorization, omp)
\tableofcontents

\chapter{Introduction}
There are many techniques for improving performance of computations. Many need special or a huge
amount of hardware and as such are expensive. Vectorisation is readily available since TODO: DATE
and most if not all modern hardware supports vectorization. If code is written to allow this form
of parallelism the performance gain gain can be, depending on the type used in the calculation, as
huge as 400\% and more. As such every programmer should have a
basic understanding of what vectorization is, what it can do and how to use it to
improve performance. The goal of this report is to give a basic understanding of vectorization and
to be used as a introduction to writing vectorized code. Most of the examples are for GNU's GCC
compiler but differences between compilers in regards to vectorization are detailed as well.
In this context this report includes the different ways vectorization can be applied and the pros
and cons for each technique.

%_____________________________________________________________________________________________________
\chapter{Instruction Sets}
    \section{MMX}
%_ _ _ _ _ _ _ _ _ _ _ _ _ _ _ _ _ _ _ _ _ _ _ _ _ _ _ _ _ _ _ _ _ _ _ _ _ _ _ _ _ _ _ _ _ _ _ _ _ _ _
    \section{AVX}
%_ _ _ _ _ _ _ _ _ _ _ _ _ _ _ _ _ _ _ _ _ _ _ _ _ _ _ _ _ _ _ _ _ _ _ _ _ _ _ _ _ _ _ _ _ _ _ _ _ _ _
    \section{SSE}
%_ _ _ _ _ _ _ _ _ _ _ _ _ _ _ _ _ _ _ _ _ _ _ _ _ _ _ _ _ _ _ _ _ _ _ _ _ _ _ _ _ _ _ _ _ _ _ _ _ _ _
    \section{AVX2}
%_ _ _ _ _ _ _ _ _ _ _ _ _ _ _ _ _ _ _ _ _ _ _ _ _ _ _ _ _ _ _ _ _ _ _ _ _ _ _ _ _ _ _ _ _ _ _ _ _ _ _
    \section{SSE2}
%_ _ _ _ _ _ _ _ _ _ _ _ _ _ _ _ _ _ _ _ _ _ _ _ _ _ _ _ _ _ _ _ _ _ _ _ _ _ _ _ _ _ _ _ _ _ _ _ _ _ _
    \section{Architectures and Sets}
%_ _ _ _ _ _ _ _ _ _ _ _ _ _ _ _ _ _ _ _ _ _ _ _ _ _ _ _ _ _ _ _ _ _ _ _ _ _ _ _ _ _ _ _ _ _ _ _ _ _ _
\chapter{Vectorization CPU components}
In the Chapter prior to this it was shown what the different instruction sets allow to be done. But
the question how it is possible that a CPU can calculate 2 - 16 calculations at the same time was
not answered. In this section we will tackle the answer to that question and elaborate the
different components present in every modern CPU that make vectorization possible.
There are three components that the extended instruction sets use: The vector-unit, the
vector-registers and the bus that connects regular memory with the just mentioned
vector-components. Details on them are in the following sections.
\section{Vector Units}

%_ _ _ _ _ _ _ _ _ _ _ _ _ _ _ _ _ _ _ _ _ _ _ _ _ _ _ _ _ _ _ _ _ _ _ _ _ _ _ _ _ _ _ _ _ _ _ _ _ _ _
\section{Vector Registers}

\section{TODO: GIVE NAME, the bus}
%_ _ _ _ _ _ _ _ _ _ _ _ _ _ _ _ _ _ _ _ _ _ _ _ _ _ _ _ _ _ _ _ _ _ _ _ _ _ _ _ _ _ _ _ _ _ _ _ _ _ _

%_____________________________________________________________________________________________________
\chapter{Requirements for Vectorization}
    %continues data
    %continues data access
    %no if block
%_____________________________________________________________________________________________________
\chapter{Auto vectorization}
    \section{Compiler differences}
%_ _ _ _ _ _ _ _ _ _ _ _ _ _ _ _ _ _ _ _ _ _ _ _ _ _ _ _ _ _ _ _ _ _ _ _ _ _ _ _ _ _ _ _ _ _ _ _ _ _ _

%_____________________________________________________________________________________________________
\chapter{Applying Vectorization manualy}
    \section{OMP}
%_ _ _ _ _ _ _ _ _ _ _ _ _ _ _ _ _ _ _ _ _ _ _ _ _ _ _ _ _ _ _ _ _ _ _ _ _ _ _ _ _ _ _ _ _ _ _ _ _ _ _
    \section{intrinsics}
%_ _ _ _ _ _ _ _ _ _ _ _ _ _ _ _ _ _ _ _ _ _ _ _ _ _ _ _ _ _ _ _ _ _ _ _ _ _ _ _ _ _ _ _ _ _ _ _ _ _ _

%_____________________________________________________________________________________________________
\chapter{Resolving issues}
%_____________________________________________________________________________________________________
\chapter{Conclusion}
\label{Conclusion}

Lorem ipsum dolor sit amet, consetetur sadipscing elitr, sed diam nonumy eirmod tempor invidunt ut labore et dolore magna aliquyam erat, sed diam voluptua.
At vero eos et accusam et justo duo dolores et ea rebum.
Stet clita kasd gubergren, no sea takimata sanctus est Lorem ipsum dolor sit amet.

\bibliographystyle{alpha}
\bibliography{literatur}

\appendix
\appendixpage

\end{document}
