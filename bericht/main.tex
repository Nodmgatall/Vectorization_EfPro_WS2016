\documentclass[
	12pt,
	a4paper,
	BCOR10mm,
	%chapterprefix,
	DIV14,
	headsepline,
	usegeometry,
	%twoside,
	%openright
]{scrreprt}

    \pdfminorversion=6

\KOMAoptions{
	listof=totoc,
	bibliography=totoc,
	index=totoc
}
\usepackage[T1]{fontenc}
\usepackage[utf8]{inputenc}
\usepackage{minted}
    \renewcommand\listingscaption{CodeSnippet}
\usepackage{lmodern}
\usepackage{pgfplots}
\usetikzlibrary{pgfplots.groupplots}
\usepackage{filecontents}
\usepackage[ngerman,english]{babel}
\usepackage[toc]{appendix}
\usepackage{color}
\usepackage{eurosym}
\usepackage{fancyhdr}
\usepackage{geometry}
\usepackage{graphicx}
\usepackage[htt]{hyphenat}
\usepackage{listings}
\usepackage{lstautogobble}
\usepackage{microtype}
\usepackage[list=true,hypcap=true]{subcaption}
\usepackage{textcomp}
\usepackage{units}
\usepackage{csvsimple}
\usepackage{varioref}
\usepackage[hidelinks]{hyperref}
\usepackage[capitalise,noabbrev]{cleveref}
\usepackage{url}

\definecolor{uhhred}{cmyk}{0,1,1,0}

\lstset{
	basicstyle=\ttfamily,
	frame=single,
	numbers=left,
	language=C,
	breaklines=true,
	breakatwhitespace=true,
	postbreak=\hbox{$\hookrightarrow$ },
	showstringspaces=false,
	autogobble=true,
	upquote=true,
	tabsize=4,
	captionpos=b,
	morekeywords={int8_t,uint8_t,int16_t,uint16_t,int32_t,uint32_t,int64_t,uint64_t,size_t,ssize_t,off_t,intptr_t,uintptr_t,mode_t}
}

\makeatletter
\renewcommand*{\lstlistlistingname}{List of Listings}
\makeatother

\begin{document}

\newgeometry{left=2cm, top=2cm, right=2cm, bottom=2cm}

\begin{titlepage}
	\includegraphics[width=0.5\textwidth]{UHH-Logo_2010_Farbe_CMYK}

	\begin{center}
		{\Large \textcolor{uhhred}{\textbf{Report}}\par}

		\vspace{1cm}

		{\titlefont\huge Vectorization \par}

		\vspace{1cm}

		{\large vorgelegt von\par}

		\vspace{0.5cm}

		{\large Vorname Nachname\par}
	\end{center}

	\vfill

	{\large\noindent\begin{tabular}{l}
		Fakultät Informatik und Naturwissenschaften\\
		Fachbereich Informatik\\
		Arbeitsbereich Wissenschaftliches Rechnen
	\end{tabular}\par}

	\vspace{1cm}

	{\large\noindent\begin{tabular}{ll}
		Studiengang:    &  SSE\\
		Matrikelnummer: & 6420331\\
		\\
		Betreuer:       & Nabeeh Jumah \\
		\\
		Hamburg, 2017-03-25
	\end{tabular}\par}
\end{titlepage}

\restoregeometry

\chapter*{Abstract}

\thispagestyle{empty}

This report discusses the basics of vectorization. It contains basic examples as well as some problems a programmer will encounter when working with
vectorization. It includes some the different instruction sets as well as differences between
compilers. The goal of this is to give a robust basic knowledge of the principles of
vectorization and to give an introduction on how to apply this form of optimization.
%TODO: add: different approaches (intinsics, auto vectorization, omp)
\tableofcontents

\chapter{Introduction}
There are many techniques for improving performance of computations. Many need special or a huge
amount of hardware and as such are expensive. Vectorisation is readily available since TODO: DATE
and most if not all modern hardware supports vectorization. If code is written to allow this form
of parallelism the performance gain gain can be, depending on the type used in the calculation, as
huge as 400\% and more. As such every programmer should have a
basic understanding of what vectorization is, what it can do and how to use it to
improve performance. The goal of this report is to give a basic understanding of vectorization and
to be used as a introduction to writing vectorized code. Most of the examples are for GNU's GCC
compiler but differences between compilers in regards to vectorization are detailed as well.
In this context this report includes the different ways vectorization can be applied and the pros
and cons for each technique.

%_____________________________________________________________________________________________________
\chapter{Instruction Sets}
Over the years several single instruction multiple data instruction sets(SIMD) were released.
In this chapter we will iterate over most of them and show the extensions each release featured.
    \section{MMX}
    MMX was introduced 1997 by Intel together with the P5-based Pentium processors. It defines 8
    64-bit wide vector register named MM0 to MM7 and the operations needed to use those. MXX offers
    only integer operations.
    For compatibility reasons the registers are nothing more than aliases for the x87 FPU
    registers.
    This meant that the programmer needed to
    decide in which mode his application should run since switching between floating point
    mode and integer mode was time consuming at best. \cite{MMX_WIK}
%_ _ _ _ _ _ _ _ _ _ _ _ _ _ _ _ _ _ _ _ _ _ _ _ _ _ _ _ _ _ _ _ _ _ _ _ _ _ _ _ _ _ _ _ _ _ _ _ _ _ _
    \section{SSE}
    SSE introduced special independent vector registers called
    XMM registers. Those registers now had a 128 bit size and were used for the with SSE introduced
    capability to use floating point operations. Integer operations were sill using the MMX
    registers. Earlier versions of SSE had 8 XMM registers and later, starting with the Intel
    64 architecture, additional 8 XMM registers were added although these are only accessible in
    64-bit operating mode. 
    This instruction set added 70 new instructions to the already present MXX instructions.
    SSE was first introduced with the pentium III 
%_ _ _ _ _ _ _ _ _ _ _ _ _ _ _ _ _ _ _ _ _ _ _ _ _ _ _ _ _ _ _ _ _ _ _ _ _ _ _ _ _ _ _ _ _ _ _ _ _ _ _
%_ _ _ _ _ _ _ _ _ _ _ _ _ _ _ _ _ _ _ _ _ _ _ _ _ _ _ _ _ _ _ _ _ _ _ _ _ _ _ _ _ _ _ _ _ _ _ _ _ _ _
    \section{SSE2}
    With this instruction set the problems of mode switching when using SIMD with integer
    operations were circumvented by implementing most SEE2 integer instructions to use the in SSE
    introduced XMM registers instead of the MXX registers. This increased performance significantly
    as the newer XMM registers were wider since they allowed to load and process two times more data.
    While the performance was increased through using the XMM registers the precision was lowered
    since FPU registers offered a higher precision than the XMM registers.
    Pentium 4 2001.
    \section{SSE3}
    SSE3 added the capability to work in a register. Meaning that with this addition it is possible
    to add and subtract values stored in a register. Further a method for converting floating point
    variables to integer variables.
%_ _ _ _ _ _ _ _ _ _ _ _ _ _ _ _ _ _ _ _ _ _ _ _ _ _ _ _ _ _ _ _ _ _ _ _ _ _ _ _ _ _ _ _ _ _ _ _ _ _ _
    \section{AVX/AVX2/AVX-512}
    With AVX the width of the vector-registers was enlarged to 256bit and the registers were
    renamed to YMM0 to YMM15. AVX also added 3 operand simd allowing ${a = a + b}$ type operations to
    be changed to ${c = a + b}$ so that the source variables $a$ and $b$ can be used in further
    calculations without changing the result. AVX was introduced together with the Sandy Bridge
    architecture in 2008.\newline
    AVX2 expanded most integer commands to 256 bits.\newline
    AVX-512 later expanded the register size to 512 bits.\cite{AVX-AVX2-AVX512}

%_ _ _ _ _ _ _ _ _ _ _ _ _ _ _ _ _ _ _ _ _ _ _ _ _ _ _ _ _ _ _ _ _ _ _ _ _ _ _ _ _ _ _ _ _ _ _ _ _ _ _
%_ _ _ _ _ _ _ _ _ _ _ _ _ _ _ _ _ _ _ _ _ _ _ _ _ _ _ _ _ _ _ _ _ _ _ _ _ _ _ _ _ _ _ _ _ _ _ _ _ _ _
\chapter{Vectorization CPU components}
In the Chapter prior to this it was shown what the different instruction sets allow to be done. But
the question how a CPU can calculate 2 - 16 calculations at the same time was
not answered. In this section we will tackle the answer to that question and elaborate the
different components present in every modern CPU that make vectorization possible.
There are three components that the extended instruction sets use: The vector-unit, the
vector-registers and the bus that connects regular memory with the just mentioned
vector-components. Details on them are in the following sections. \cite{IntelSIMDdef}
\section{Vector Units}
Vector-Units are specially designed computation units which can compute multiple operations at
once. Where a regular unit has one input of width 32bit or 64bit (depending on architecture)
vector-units have wider input ranging from 128(sse) to 512 (AVX-512). Using SSE the table
\ref{table:tac} shows how many computations can be done at once with given type.
\begin{table}[ht]
    \centering
        \begin{tabular}{c|c|c|c|c}
            Type      & MMX & SSE & AVX & AVX-512\\
            \hline
            short int & 8   & 16  & 32  & 64 \\
            \hline
            int       & 2   & 4   & 8   & 16 \\
            \hline
            float     & -   & 4   & 8   & 16 \\
            \hline
            double    & -   & 2   & 4   & 8\\
        \end{tabular} 
            \caption{type and \#calculations}
        \label{table:tac}
    \end{table}
    %_ _ _ _ _ _ _ _ _ _ _ _ _ _ _ _ _ _ _ _ _ _ _ _ _ _ _ _ _ _ _ _ _ _ _ _ _ _ _ _ _ _ _ _ _ _ _ _ _ _ _
\section{Vector Registers}
Vector registers have the same width as the vector-unit's in- and output. In these registers all
data that is meant to be processed through the vector-units is loaded to and from. When using
vector-units these are used instead of the scalar registers.

%_ _ _ _ _ _ _ _ _ _ _ _ _ _ _ _ _ _ _ _ _ _ _ _ _ _ _ _ _ _ _ _ _ _ _ _ _ _ _ _ _ _ _ _ _ _ _ _ _ _ _

%_____________________________________________________________________________________________________
\chapter{Requirements for Vectorization}
In the Chapter before we introduced  the hardware components for vectorization. Through the way
these work there are some restrictions on what code can be vectorized. In this section those
restrictions which can hinder or even make vectorization impossible will be introduced.
\section{Coherent data storage}
For vectorization to work best the data on which is operated should be stored in continues arrays.
If stored properly data can be loaded and stored in chunks while using only one instruction. As
such the approach of using objects of arrays instead of arrays of objects will result in better
vectorization in most cases. 
\begin{figure}[h]
    \centering
    \begin{minipage}[t]{.45\textwidth}
        \begin{minted}{cpp}
struct vec3d{
    type x;
    type y;
    type z;
}
std::vector<vec3d> vec3d_objs;
sum_all_x(vec3d_objs);
        \end{minted}

        \includegraphics[width=0.77\textwidth]{loading_obj_var.png}
        \caption{using array of objects}
    \end{minipage}
    \begin{minipage}[t]{.45\textwidth}
        \begin{minted}{cpp}
struct vec3ds{
    std::vector<type> x_elements;
    std::vector<type> y_elements;
    std::vector<type> z_elements;
}
vec3ds my_vectors;
my_vectors.sum_all_x();
        \end{minted}
        \includegraphics[width=0.77\textwidth]{loading_simd.png}
    \caption{TODO}
    \label{fig:obj_load}
    \end{minipage}
\end{figure} 
Although there is the option to use a simd gather instruction which will load the different
variables from figure \ref{fig:obj_load} faster than fetching the variables element wise the
restructuring of the data layout in combination with a simple simd packed load will be faster.
\section{alignment}
\includegraphics[width=0.77\textwidth]{AlignmentVisualisation.png}
\section{aliasing}
Aliasing can cause functions to perform different when vectorized. The compiler needs to check if
two pointer are references to values in the same memory block. This can hinder the vectorization or
even make it impossible. The developer need to assure the compiler that no pointer is an alias
inside the to be vectorized function.
\section{Function calls}
Non inlined function calls can not be vectorized. And if profitable should be moved into their own
loop. \cite{ExternalFunctionSource}

\begin{figure}[h]
    \centering
    \begin{minipage}[t]{.45\textwidth}
        \begin{minted}{cpp}
for(int i = 0; i < SIZE; i++){
    dest[i] = vec1[i] + vec2[i];
    dest[i] = pow(dest[i]);
    dest[i] = dest[i] / 1.4;
}
        \end{minted}

        \caption{using array of objects}
    \end{minipage}
    \begin{minipage}[t]{.45\textwidth}
        \begin{minted}{cpp}
for(int i = 0; i < SIZE; i++){
    dest[i] = vec1[i] + vec2[i];
}
for(int i = 0; i < SIZE; i++){
    dest[i] = pow(dest[i]);
}
for(int i = 0; i < SIZE; i++){
    dest[i] = dest[i] / 1.4;
}
        \end{minted}
    \caption{TODO}
    \label{fig:obj_load}
    \end{minipage}
\end{figure} 
If moved the first and the second loop will be vectorized. 

\section{Branching}
Branching through switch case statements or if-else blocks are not vectorizable. There are some
exceptions e.g. a very simple if statement such as:
        \begin{minted}{cpp}
            bool b[16] = {false, false, ...}:
            int interger[16];
            for(int i = 0; i < 16; i++){
            if(interger[i] < 10) { b[i] = true;}
            }
        \end{minted}
%_____________________________________________________________________________________________________
\chapter{Auto vectorization}
%_ _ _ _ _ _ _ _ _ _ _ _ _ _ _ _ _ _ _ _ _ _ _ _ _ _ _ _ _ _ _ _ _ _ _ _ _ _ _ _ _ _ _ _ _ _ _ _ _ _ _
%_ _ _ _ _ _ _ _ _ _ _ _ _ _ _ _ _ _ _ _ _ _ _ _ _ _ _ _ _ _ _ _ _ _ _ _ _ _ _ _ _ _ _ _ _ _ _ _ _ _ _
Modern compilers offer to vectorize programs for the developer. To enable compilers to do so,
certain requirements must be met. In this chapter those requirements will be explained in context
with the GNU compiler Gcc.The default optimization flag for Gcc is -O2 which does not include 
auto vectorisation. To use the auto vectorisation either the optimization flag -O3 (or better e.g. -Ofast)
or the flag -ftree-vectorize have to be used. In some cases further flags like -fassociative-math
are required for vectorization. The following code will be a example on how to change code so that
it can be successfully vectorized.
\begin{listing}[ht]
\begin{minted}[linenos]{cpp}
void normal(type *vec1, type *vec2,
    unsigned long size, type *return_vec)
{
    for (unsigned long i = 0; i < size; i++)
    {
        return_vec[i] += vec1[i] + vec2[i];
    }
}
\end{minted}
\caption{Function without changes for vectorization}
\label{lst:code1}
\end{listing}
This function is very simple as the only thing it does is to add 2 arrays element wise and to store
the results in another array.
When compiling this with \newline
g++ -std=c++11 -fopt-info-optimized -ftree-vectorize -fopt-info-missed -O2 -funroll-loops -msse3
main.cpp we get the following output:
\begin{minted}[fontsize=\footnotesize]{bash}
g++ -c  normalvec.cpp -msse3 -O2 -funroll-loops -ftree-vectorize-fopt-info-vec-optimized
normal.cpp:4:33: note: loop vectorized
normal.cpp:4:33: note: loop versioned for vectorization because of possible aliasing
normal.cpp:4:33: note: loop peeled for vectorization to enhance alignment
\end{minted}
Since the function is very simple the gcc 6.3 compiler is able do vectorize this to be even better
than what will be shown in the following sections. For the sake of explaining what would be needed
to write a more complex vectorizable function we will use the function without any vectorization enabled to
highlight the potential difference vectorization can make. In graph \ref{fig:graphNorVecnor} we see the run times of
the unvectorized function in comparison with the completely auto vectorized function. 
\begin{figure}[ht]
    \begin{tikzpicture}[scale=0.66]

    \begin{axis}[legend pos=outer north east,
        xlabel=\#Entries,
        ylabel=runtime sum
        ]
             \addplot table [y=normal       , x=rowname , col sep=comma]{src/result_new_sse3_O2_unroll.txt};
            \addlegendentry{normal }
  \addplot table [y=normalvec     , x=rowname , col sep=comma]{src/result_new_sse3_O2_unroll.txt};
            \addlegendentry{normalvec }

        \end{axis}
    \end{tikzpicture}
    \caption{unvectorized function}
    \label{fig:graphNorVecnor}
\end{figure}
In the following section we will show the steps to vectorize the function manually. It is to note
that vectorizing manually disallows the compiler to vectorize as he would normally do. Thus the
results of each step of vectorizing the function is worse than doing nothing. All steps combined
are as good as if nothing had been done. This section is only for highlighting the difference in
the different steps and to explain the different concepts behind what makes vectorization possible
for more complex functions.



%_ _ _ _ _ _ _ _ _ _ _ _ _ _ _ _ _ _ _ _ _ _ _ _ _ _ _ _ _ _ _ _ _ _ _ _ _ _ _ _ _ _ _ _ _ _ _ _ _ _ _
\section{No aliases}
Vectorization does not work when arrays on which the calculations should be performed are
overlapping or are aliases of each other. In the earlier shown function (CodeSnippet:
[\ref{lst:code1}]) the compiler would not know if the given pointer point at the same or to
a given sub part of memory. To let the compiler know that they are not overlapping the keyword
restrict can be used (CodeSnippet\ref{lst:code2}: line:1). This keyword should only be used if the
developer knows for certain that the given pointer will never refer to the same memory block.
Otherwise the results of the calculations can (and probably will) be wrong. If used correctly the
restrict keyword or any other form of assuring the compiler that there are no overlaps will result
in a better vectorization as extra checks if overlaps are present will not be needed.
\begin{listing}
\begin{minted}[fontsize=\footnotesize]{cpp}
void no_alias(type *__restrict__ vec1,
    type *__restrict__ vec2, unsigned long size,
              type *__restrict__ return_vec)
{
    for (unsigned long i = 0; i < size; i++) {
        return_vec[i] += vec1[i] + vec2[i];
    }
}
\end{minted}
\caption{Function without changes for vectorization}
\label{lst:code2}
\end{listing}

When we compile this function with the same parameters as stated in the beginning of this chapter
we get: 
\begin{minted}[fontsize=\footnotesize]{bash}
g++ -c  no_alias.cpp -msse3 -O1 -funroll-loops -ftree-vectorize-fopt-info-vec-optimized
no_alias.cpp:4:33: note: loop vectorized
no_alias.cpp:4:33: note: loop peeled for vectorization to enhance alignment
\end{minted}
Here we see that now the program does no longer need to check for aliases.
The following graph \ref{tik:uvVsNa} shows the difference between the unvectorized function and the function where
we assured the compiler that the arrays vec1 and vec2 are distinct memory blocks.

\begin{figure}[ht]
  \begin{tikzpicture}[scale=0.66]

        \begin{axis}[legend pos=outer north east,
            xlabel=\#Entries,
            ylabel=runtime sum
            ]
                 \addplot table [y=normal      , x=rowname , col sep=comma]{src/result_new_sse3_O1_unroll.txt};
                \addlegendentry{normal }
                \addplot table [y=noalias      , x=rowname , col sep=comma]{src/result_new_sse3_O1_unroll.txt};
                \addlegendentry{noalias}
            \end{axis}
        \end{tikzpicture}
        \caption{unvectorized function}
        \label{tik:uvVsNa}
    \end{figure}


%_ _ _ _ _ _ _ _ _ _ _ _ _ _ _ _ _ _ _ _ _ _ _ _ _ _ _ _ _ _ _ _ _ _ _ _ _ _ _ _ _ _ _ _ _ _ _ _ _ _ _
\section{Alignment}
Alignment allows for packaged data which can be loaded in one instruction.
\subsection{Scalar}
\subsection{Packaged}

\begin{figure}[ht]
    \begin{tikzpicture}[scale=0.66]

        \begin{axis}[legend pos=outer north east,
            xlabel=\#Entries,
            ylabel=runtime sum
            ]
                 \addplot table [y=normal , x=rowname , col sep=comma]{src/result_new_sse3_O1_unroll.txt};
                \addlegendentry{normal }
                \addplot table [y=aligned , x=rowname , col sep=comma]{src/result_new_sse3_O1_unroll.txt};
                \addlegendentry{aligned}

            \end{axis}
        \end{tikzpicture}
        \caption{unvectorized function}
    \end{figure}
\begin{minted}[fontsize=\footnotesize]{bash}
g++ -c  aligned.cpp -msse3 -O2 -funroll-loops -ftree-vectorize-fopt-info-vec-optimized
normal.cpp:4:33: note: loop vectorized
normal.cpp:4:33: note: loop versioned for vectorization because of possible aliasing
\end{minted}

%_ _ _ _ _ _ _ _ _ _ _ _ _ _ _ _ _ _ _ _ _ _ _ _ _ _ _ _ _ _ _ _ _ _ _ _ _ _ _ _ _ _ _ _ _ _ _ _ _ _ _
\section{Alignment and Restrict combined}

\begin{minted}[fontsize=\footnotesize]{bash}
g++ -c  vectorizable.cpp -msse3 -O2 -funroll-loops -ftree-vectorize-fopt-info-vec-optimized
normal.cpp:4:33: note: loop vectorized
normal.cpp:4:33: note: loop versioned for vectorization because of possible aliasing
\end{minted}
\begin{figure}[ht]
  \begin{tikzpicture}[scale=0.66]

        \begin{axis}[legend pos=outer north east,
            xlabel=\#Entries,
            ylabel=runtime sum
            ]
            %     \addplot table [y=normal      , x=rowname , col sep=comma]{src/result_sse3_sse3_O2_unroll.txt};
            %    \addlegendentry{normal }
            %    \addplot table [y=aligned      , x=rowname , col sep=comma]{src/result_sse3_sse3_O2_unroll.txt};
            %    \addlegendentry{aligned}
            %    \addplot table [y=noalias      , x=rowname , col sep=comma]{src/result_sse3_sse3_O2_unroll.txt};
            %    \addlegendentry{noalias}
            %    \addplot table [y=vectorizable , x=rowname , col sep=comma]{src/result_sse3_sse3_O2_unroll.txt};
            %    \addlegendentry{vectorizable}

            %\addplot table [y=normal      , x=rowname , col sep=comma]{src/result_new_sse3_O2_unroll.txt};
            %    \addlegendentry{normal }
                \addplot table [y=aligned      , x=rowname , col sep=comma]{src/result_new_sse3_O2_unroll.txt};
                \addlegendentry{aligned}
                \addplot table [y=noalias      , x=rowname , col sep=comma]{src/result_new_sse3_O2_unroll.txt};
                \addlegendentry{noalias}
                \addplot table [y=vectorizable , x=rowname , col sep=comma]{src/result_new_sse3_O2_unroll.txt};
                \addlegendentry{vectorizable}
 \addplot table [y=normalvec     , x=rowname , col sep=comma]{src/result_new_sse3_O2_unroll.txt};
            \addlegendentry{normalvec }


            \end{axis}
        \end{tikzpicture}
        \caption{unvectorized function}
    \end{figure}
\begin{figure}[ht]
    \begin{tikzpicture}[scale=0.66]
        \begin{axis}[legend pos=outer north east,
            xlabel=\#Entries,
            ylabel=runtime sum%,
              %ymin=2.5, xmin=20000,
                %xmax=163000, ymax=6,
                %restrict y to domain=2.5:6
                ]

            \addplot table [y expr=\thisrow{normal}/\thisrow{aligned}
                , x=rowname , col sep=comma]{src/result_new_sse3_O2_unroll.txt};
            \addlegendentry{aligned}
            \addplot table [y expr=\thisrow{normal}/\thisrow{noalias}
                , x=rowname , col sep=comma]{src/result_new_sse3_O2_unroll.txt};
            \addlegendentry{noalias}
            \addplot table [y expr=\thisrow{normal}/\thisrow{vectorizable} 
                , x=rowname , col sep=comma]{src/result_new_sse3_O2_unroll.txt};
            \addlegendentry{vectorizable}
\addplot table [y expr=\thisrow{normal}/\thisrow{normalvec} 
                , x=rowname , col sep=comma]{src/result_new_sse3_O2_unroll.txt};
            \addlegendentry{normalvec}

            \end{axis}
        \end{tikzpicture}
        \caption{stages of vectorization speedup}
    \end{figure}
%_ _ _ _ _ _ _ _ _ _ _ _ _ _ _ _ _ _ _ _ _ _ _ _ _ _ _ _ _ _ _ _ _ _ _ _ _ _ _ _ _ _ _ _ _ _ _ _ _ _ _

%_____________________________________________________________________________________________________

\chapter{Applying Vectorization manualy}
    \section{OMP}
\begin{figure}[ht]
    \begin{tikzpicture}
        \begin{axis}[legend pos=outer north east%,
                %ymin=1.0, xmin=20000,
                %xmax=163000, ymax=4,
                %restrict y to domain=1:4
            ]

\addplot table [y expr=\thisrow{normal}/\thisrow{normalvec}    , x=rowname , col sep=comma]{src/result_new_sse3_O2_unroll.txt};
                \addlegendentry{normalvec }

                \addplot table [y expr=\thisrow{normal}/\thisrow{aligned}    , x=rowname , col sep=comma]{src/result_new_sse3_O2_unroll.txt};
                \addlegendentry{aligned }

                \addplot table [y expr=\thisrow{normal}/\thisrow{noalias}      , x=rowname , col sep=comma]{src/result_new_sse3_O2_unroll.txt};
                \addlegendentry{noalias }

                \addplot table [y expr=\thisrow{normal}/\thisrow{vectorizable} , x=rowname , col sep=comma]{src/result_new_sse3_O2_unroll.txt};
                \addlegendentry{vectorizable }

                \addplot table [y expr=\thisrow{normal}/\thisrow{omp}        , x=rowname , col sep=comma]{src/result_new_sse3_O2_unroll.txt};
                \addlegendentry{omp }
            \end{axis}
        \end{tikzpicture}
        \caption{TODO NEEDS A NAME}
    \end{figure}




%_ _ _ _ _ _ _ _ _ _ _ _ _ _ _ _ _ _ _ _ _ _ _ _ _ _ _ _ _ _ _ _ _ _ _ _ _ _ _ _ _ _ _ _ _ _ _ _ _ _ _
    \section{intrinsics}
%_ _ _ _ _ _ _ _ _ _ _ _ _ _ _ _ _ _ _ _ _ _ _ _ _ _ _ _ _ _ _ _ _ _ _ _ _ _ _ _ _ _ _ _ _ _ _ _ _ _ _
 

%__________________________________________________________________________NEWSTUFF


%_____________________________________________________________________________________________________
%_____________________________________________________________________________________________________
\chapter{Conclusion}
\label{Conclusion}

Lorem ipsum dolor sit amet, consetetur sadipscing elitr, sed diam nonumy eirmod tempor invidunt ut labore et dolore magna aliquyam erat, sed diam voluptua.
At vero eos et accusam et justo duo dolores et ea rebum.
Stet clita kasd gubergren, no sea takimata sanctus est Lorem ipsum dolor sit amet.

\bibliography{literatur}
\bibliographystyle{ieeetr}

\begin{figure}[ht]
    \begin{tikzpicture}[scale=0.66]

        \begin{axis}[legend pos=outer north east,
            xlabel=\#Entries,
            ylabel=runtime sum
            ] 
                \addplot table [y=normalvec , x=rowname , col sep=comma]{src/result_new_sse3_O1_unroll.txt};
                \addlegendentry{normalvec }
                \addplot table [y=normal , x=rowname , col sep=comma]{src/result_new_sse3_O1_unroll.txt};
                \addlegendentry{normal }
                \addplot table [y=aligned , x=rowname , col sep=comma]{src/result_new_sse3_O1_unroll.txt};
                \addlegendentry{aligned}
                \addplot table [y=noalias , x=rowname , col sep=comma]{src/result_new_sse3_O1_unroll.txt};
                \addlegendentry{noalias}

        \end{axis}
    \end{tikzpicture}
    \caption{unvectorized function}
\end{figure}
\begin{figure}[ht]
    \begin{tikzpicture}[scale=0.66]

        \begin{axis}[legend pos=outer north east,
            xlabel=\#Entries,
            ylabel=runtime sum
            ] 
            \addplot table [y=normalvec , x=rowname , col sep=comma]{src/result_new_sse3_O1_unroll.txt};
            \addlegendentry{normalvec }
            \addplot table [y=normalvec , x=rowname , col sep=comma]{src/result_new_sse3_O1_nounroll.txt};
            \addlegendentry{normalvec }
            \addplot table [y=normalvec , x=rowname , col sep=comma]{src/result_new_sse3_O2_unroll.txt};
            \addlegendentry{normalvec }
            \addplot table [y=normalvec , x=rowname , col sep=comma]{src/result_new_sse3_O2_nounroll.txt};
            \addlegendentry{normalvec }
        \end{axis}
    \end{tikzpicture}
    \caption{unvectorized function}
\end{figure}
\end{document}

